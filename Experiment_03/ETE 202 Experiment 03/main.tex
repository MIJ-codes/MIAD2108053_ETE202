\documentclass[a4paper, 12pt]{extarticle}
\usepackage[utf8]{inputenc}
\usepackage[margin = 1.2in]{geometry}
\usepackage{fancyhdr}

\usepackage{graphicx}
\usepackage{caption}
\usepackage{subcaption}
\usepackage{float}
\usepackage{booktabs}
\usepackage{gensymb}

\pagestyle{fancy} % to change the page style and get that overline 
\fancyhead[]{} % to delete the header over the overline
\fancyfoot[]{} % to delelte the page number from the bottom of the page
\fancyhead[R]{\thepage} % to put the page number in the right side of the overline.
\geometry{headsep=20pt} % to make a gap of 20 pt between overline and the text

\begin{document}

\newgeometry{margin = 1.2in, left = 1.5in, bottom = 0cm}
\section{Objective:}
    \begin{itemize}
        \item To determine the voltage, current and power gain of common emitter amplifier circuit.
        \item To determine the voltage, current and power gain of common base amplifier circuit.
        \item To determine the voltage, current and power gain of common collector amplifier circuit.
    \end{itemize}
\section{Apparatus:}
\begin{minipage}[t]{0.45\linewidth}
      \begin{itemize}
        \item Transistors
        \item Resistors
        \item Variable resistors
        \item Multimeter
      \end{itemize}
    \end{minipage}
    \hfill
    \begin{minipage}[t]{0.45\linewidth}
      \begin{itemize}
        \item Breadboard
        \item Wires
        \item DC power supply unit
        \item Function generator
      \end{itemize}
    \end{minipage}
\section{Circuit diagram:}
    \begin{figure}[htbp]
    \centering
        \begin{subfigure}[h]{0.45\textwidth}
        \centering
            \includegraphics[scale = 0.4]{Figures/Common Emitter Configuration Amplifier.pdf}
            \caption{Common emitter amplifier circuit.}
            \label{fig:Common emitter amplifier circuit}
        \end{subfigure}
        \hfill
        \begin{subfigure}[h]{0.45\textwidth}
        \centering
            \includegraphics[scale = 0.4]{Figures/Common Base Configuration Amplifier.pdf}
            \caption{Common base amplifier circuit.}
            \label{fig:Common base amplifier circuit}
        \end{subfigure}
        \\
        \begin{subfigure}[h]{\textwidth}
        \centering
            \includegraphics[scale = 0.4]{Figures/Common Collector Configuration Amplifier.pdf}
            \caption{Common collector amplifier circuit.}
            \label{fig:Common collector amplifier circuit}
        \end{subfigure}
        \caption{Transistor amplifier circuits.}
        \label{fig:1}
    \end{figure}

\newpage
\restoregeometry
\newgeometry{margin = 1.2in, right = 1.5in, bottom = 0cm}
\section{Result analysis:}
\subsection{Common emitter amplifier:}
In common emitter amplifier circuits both voltage gain and current gain occurs.The voltage gain is higher than than the current gain. The detailed datas are in table \ref{tab:Datas of common emitter amplifier}. The amplified output voltage is 180\degree out pf phase with each other. It can be observed from figure \ref{fig:Common Emitter Amplifier Graphs.}
% Table generated by Excel2LaTeX from sheet 'Sheet1'
\begin{table}[htbp]
  \centering
  \caption{Data of common emitter amplifier.}
    \begin{tabular}{|c|c|c|c|}
    \toprule
    \multicolumn{4}{|c|}{Common Emitter Amplifier} \\
    \midrule
          & Practical data & Simulated data & Practical data \\
    \midrule
    Vin (mV) & 100   & 100   & 30 \\
    \midrule
    Iin (µA) & 22    & 22    & 6.1 \\
    \midrule
    Vout (V) & 2.244 & 2.92  & 0.8 \\
    \midrule
    Iout (µA) & 222   & 215   & 56.7 \\
    \midrule
    AV    & 22.44 & 29.2  & 26.66666667 \\
    \midrule
    Ai    & 10.09090909 & 9.772727273 & 9.295081967 \\
    \midrule
    AP    & 226.44 & 285.3636364 & 247.8688525 \\
    \bottomrule
    \end{tabular}%
  \label{tab:Datas of common emitter amplifier}%
\end{table}%
\\
\begin{figure}[htbp]
    \centering
        \begin{subfigure}[h]{0.48\textwidth}
        \centering
            \includegraphics[scale = 0.47]{Figures/Commo Emitter Oscilloscope.pdf}
            \caption{Simulation graph.}
            \label{fig:Common Emitter Amplifier Simulation Graph}
        \end{subfigure}
        \hfill
        \begin{subfigure}[h]{0.48\textwidth}
        \centering
            \includegraphics[scale = 0.2]{Figures/Common Emitter Amplifier Experimental Graph.jpeg}
            \caption{Experimental graph.}
            \label{fig:Common Emitter Amplifier Experimental Graph}
        \end{subfigure}
        \caption{Common emitter amplifier graphs.}
        \label{fig:Common Emitter Amplifier Graphs.}
\end{figure}

\newpage
\restoregeometry
\newgeometry{margin = 1.2in, bottom = 0in, left = 1.5in}
\subsection{Common collector amplifier:}
In common collector amplifier the output voltage is the same as input voltage. Thus in this circuit there is no voltage gain but there is current gain. This output is also not out of phase. The are given in table \ref{tab:Datas of common collector amplifier} and figure \ref{fig:Common Collector Amplifier Graphs.}. 
% Table generated by Excel2LaTeX from sheet 'Sheet1'
\begin{table}[htbp]
  \centering
  \caption{Data of common collector amplifier.}
    \begin{tabular}{|c|c|c|c|}
    \toprule
    \multicolumn{4}{|c|}{Common Collector Amplifier} \\
    \midrule
          & Theoretical data & Simulated data & Practical data \\
    \midrule
    Vin (V) & 2.9975 & 3     & 3 \\
    \midrule
    Iin (µA) & 263.38 & 170   & 135.56 \\
    \midrule
    Vout (V) & 2.976 & 2.95  & 2.35 \\
    \midrule
    Iout (mA) & 2.976 & 3.97  & 4.37 \\
    \midrule
    AV    & 0.992827356 & 0.983333333 & 0.783333333 \\
    \midrule
    Ai    & 11.29926342 & 23.35294118 & 32.23664798 \\
    \midrule
    AP    & 11.21821783 & 22.96372549 & 25.25204092 \\
    \bottomrule
    \end{tabular}%
  \label{tab:Datas of common collector amplifier}%
\end{table}%

\begin{figure}[htbp]
    \centering
        \begin{subfigure}[h]{0.48\textwidth}
        \centering
            \includegraphics[scale = 0.47]{Figures/Common Collector Oscilloscope.pdf}
            \caption{Simulation graph.}
            \label{fig:Common Collector Amplifier Simulation Graph}
        \end{subfigure}
        \hfill
        \begin{subfigure}[h]{0.48\textwidth}
        \centering
            \includegraphics[scale = 0.17]{Figures/Common Collector Amplifier Experimental Graph.jpg}
            \caption{Experimental graph.}
            \label{fig:Common Collector Amplifier Experimental Graph}
        \end{subfigure}
        \caption{Common collector amplifier graphs.}
        \label{fig:Common Collector Amplifier Graphs.}
\end{figure}

\newpage
\restoregeometry
\newgeometry{margin = 1.2in, right = 1.5in, bottom = 0cm}
\subsection{Common base amplifier:}
The voltage amplification in the common base amplifier is the same as common emitter amplifier but there is no phase difference. But the current gain of the circuit is almost equal to 1, means there is not current gain. It much visible from the table \ref{tab:Datas of common base amplifier} and figure \ref{fig:Common Base Amplifier Graphs.}.

% Table generated by Excel2LaTeX from sheet 'Sheet1'
\begin{table}[htbp]
  \centering
  \caption{Data of common base amplifier.}
    \begin{tabular}{|c|c|c|c|}
    \toprule
    \multicolumn{4}{|c|}{Common Base Amplifier} \\
    \midrule
          & Theoretical data & Simulated data & Practical data \\
    \midrule
    Vin (mV) & 100   & 100   & 100 \\
    \midrule
    Iin (mA) & 6.19  & 1.42  & 1.07 \\
    \midrule
    Vout (V) & 11.2039 & 2.65  & 2.18 \\
    \midrule
    Iout (mA) & 6.13  & 1.39  & 1.01 \\
    \midrule
    AV    & 11.2039 & 2.65  & 2.18 \\
    \midrule
    Ai    & 0.990306947 & 0.978873239 & 0.943925234 \\
    \midrule
    AP    & 11.0953 & 2.594014085 & 2.057757009 \\
    \bottomrule
    \end{tabular}%
  \label{tab:Datas of common base amplifier}%
\end{table}%

\begin{figure}[htbp]
    \centering
        \begin{subfigure}[h]{0.48\textwidth}
        \centering
            \includegraphics[scale = 0.47]{Figures/Commo Base Oscilloscope.pdf}
            \caption{Simulation graph.}
            \label{fig:Common Base Amplifier Simulation Graph}
        \end{subfigure}
        \hfill
        \begin{subfigure}[h]{0.48\textwidth}
        \centering
            \includegraphics[scale = 0.17]{Figures/Common Base Amplifier Experimental Graph.jpg}
            \caption{Experimental graph.}
            \label{fig:Common Base Amplifier Experimental Graph}
        \end{subfigure}
        \caption{Common base amplifier graphs.}
        \label{fig:Common Base Amplifier Graphs.}
\end{figure}

\newpage
\restoregeometry
\newgeometry{margin = 1.2in, left = 1.5in}
So from these analysis it can be concluded that the common emitter amplifier circuits have high voltage and current gain and also there is phase shift between input and output.
The common collector has no voltage gain but high current gain. That's why it's called the buffer amplifier.
The common base has high voltage gain but no current gain.Also there is no phase shift between them.
The power gain in all of these circuits is: Common emitter > common collector > common base.
\subsection{Discussion:}
\begin{enumerate}
    \item The experiment was conducted to learn about the amplification of transistors. To do this three circuits: common emitter, common base, common collector was used.
    \item The data from theory, simulation and experiment isn't same.
    \item The difference in theory was because there are resistances called source resistance ($R_S$) and input base resistance ($R_{in(base)}$) that affect the output in simulation and experiment. But in theory it's hard tho know the values of these resistances as they vary from equipment to equipment.
    \item  The voltage gain depends on $r_e$, $r_e$ depends on $I_e$ and $I_e$ changes with temperature. It's hard to control the temperature in a lab so that's why there were differences between simulation and experiment.
    \item The way to reduce errors could have been doing the experiment in a much more controlled environment. Taking account of the internal changes inside an equipment would also help reducing errors.
\end{enumerate}
\end{document}
