\documentclass[a4paper, 12pt]{extarticle}
\usepackage[utf8]{inputenc}
\usepackage[margin=1.2in]{geometry}
\usepackage{fancyhdr}
\usepackage{graphicx}
\usepackage{caption}
\usepackage{subcaption}
\usepackage{float}
\usepackage{booktabs}
\usepackage{gensymb}
\usepackage{longtable}

\pagestyle{fancy} % to change the page style and get that overline 
\fancyhead[]{} % to delete the header over the overline
\fancyfoot[]{} % to delete the page number from the bottom of the page
\fancyhead[R]{\thepage} % to put the page number in the right side of the overline.
\geometry{headsep=20pt} % to make a gap of 20 pt between overline and the text

\begin{document}

\newgeometry{margin=1.2in, left=1.5in, bottom=0cm}
\section{Objective:}
    \begin{itemize}
        \item To determine the frequency response of a band-stop filter.
        \item To measure the cutoff frequencies and observe the attenuation rate.
        \item To compare simulated and experimental results.
    \end{itemize}

\section{Apparatus:}
\begin{minipage}[t]{0.45\linewidth}
    \begin{itemize}
        \item Resistors
        \item Capacitors
        \item Wires
    \end{itemize}
\end{minipage}
\hfill
\begin{minipage}[t]{0.45\linewidth}
    \begin{itemize}
        \item Breadboard
        \item Function generator
        \item Multimeter
    \end{itemize}
\end{minipage}

\section{Circuit Diagram:}
\begin{figure}[htbp]
    \centering
    \includegraphics[scale=0.6]{Figures/Band stop latest.pdf}
    \caption{Band-stop filter circuit diagram.}
    \label{fig:Band_Stop_Filter_Circuit}
\end{figure}

\newpage
\restoregeometry
\newgeometry{margin=1.2in, left=1.5in, bottom=0cm}

\section{Result Analysis:}
\subsection{Band-Stop Filter:}
The band-stop filter is designed to attenuate signals within a specific frequency range while allowing frequencies outside this range to pass. The detailed data is shown in Table \ref{tab:Band_Stop_Filter_Data}. The cutoff frequencies are observed at the points where the output voltage drops to 70.7\% of the input voltage at the lower and upper bounds of the stop band.

% Table placeholder for band-stop filter data
% Table generated by Excel2LaTeX from sheet 'Sheet1'
\begin{longtable}{cccc}
    \caption{Band-stop filter data} \label{tab:Band_Stop_Filter_Data} \\
    \toprule
    Logarithmic Frequency & Vin (V) & Av (dB) & Av (dB) Sim \\
    \midrule
    \endfirsthead
    \caption[]{Band-stop filter data (continued)} \\
    \toprule
    Logarithmic Frequency & Vin (V) & Av (dB) & Av (dB) Sim \\
    \midrule
    \endhead
    \bottomrule
    \endfoot
    0     & 2     & -56.47817482 & -56.47817482 \\
    0.176091259 & 2     & -35.65032112 & -50.45757491 \\
    0.301029996 & 2     & -35.65032112 & -46.02059991 \\
    0.397940009 & 2     & -36.19336604 & -35.91760035 \\
    0.477121255 & 2     & -36.19336604 & -34.19930777 \\
    0.544068044 & 2     & -36.19336604 & -32.95634964 \\
    0.602059991 & 2     & -36.19336604 & -30.17276612 \\
    0.653212514 & 2     & -35.91760035 & -28.51937465 \\
    0.698970004 & 2     & -35.91760035 & -26.74484337 \\
    0.740362689 & 2     & -35.91760035 & -22.2701855 \\
    0.77815125 & 2     & -35.13923903 & -21.67092103 \\
    0.812913357 & 2     & -33.97940009 & -20.49136383 \\
    0.84509804 & 2     & -8.995432939 & -19.41232445 \\
    0.875061263 & 2     & -21.99265743 & -18.56235985 \\
    0.903089987 & 2     & -18.48906077 & -18.09661297 \\
    0.929418926 & 2     & -17.13970399 & -16.83275016 \\
    0.954242509 & 2     & -16.13750803 & -15.94478625 \\
    1     & 2     & -14.60974112 & -14.49377291 \\
    1.301029996 & 2     & -9.76233278 & -12.70898009 \\
    1.477121255 & 2     & -8.024189865 & -8.565823364 \\
    1.602059991 & 2     & -7.071925476 & -7.091554613 \\
    1.698970004 & 2     & -6.456989575 & -6.429632419 \\
    1.77815125 & 2     & -6.125461022 & -6.055413145 \\
    2     & 2     & -5.763855419 & -5.272069954 \\
    2.698970004 & 2     & -4.159173786 & -4.082399653 \\
    3     & 2     & -4.145164283 & -4.096308206 \\
    3.698970004 & 2     & -4.279175795 & -4.229562256 \\
    4     & 2     & -4.516962821 & -4.882502887 \\
    4.301029996 & 2     & -4.882502887 & -5.514482608 \\
    4.477121255 & 2     & -5.161218445 & -5.352124804 \\
    4.602059991 & 2     & -5.489934623 & -5.679933127 \\
    4.698970004 & 2     & -5.874180855 & -6.011918364 \\
    5     & 2     & -8.635965519 & -8.178707859 \\
    5.301029996 & 2     & -12.99503963 & -12.04119983 \\
    5.477121255 & 2     & -21.11034656 & -22.55687455 \\
    5.602059991 & 2     & -29.11863911 & -30.60355968 \\
    5.698970004 & 2     & -35.13923903 & -36.19336604 \\
    6     & 2     & -31.05683937 & -29.70904495 \\
    6.301029996 & 2     & -21.93820026 & -20.91514981 \\
    6.477121255 & 2     & -5.638274104 & -5.596813932 \\
    6.698970004 & 2     & -5.638274104 & -5.514482608 \\
    7     & 2     & -5.61337426 & -5.654913745 \\
    7.176091259 & 2     & -5.605090149 & -5.605090149 \\
    7.301029996 & 2     & -5.580285117 & -5.596813932 \\
    7.397940009 & 2     & -5.688288962 & -5.755426608 \\
    7.477121255 & 2     & -5.934172438 & -5.977278678 \\
    7.544068044 & 2     & -12.95634964 & -13.76492278 \\
    7.602059991 & 2     & -18.06179974 & -19.57621402 \\
    7.653212514 & 2     & -21.41162149 & -23.0980392 \\
    7.698970004 & 2     & -26.46612781 & -26.83977207 \\
    6.698970004 & 2     & -29.76233278 & -28.17870786 \\
    7.740362689 & 2     & -33.76492278 & -29.76233278 \\
    7.77815125 & 2     & -36.19336604 & -32.39577517 \\
\end{longtable}



\begin{figure}[htbp]
    \centering
    \includegraphics[scale=0.6]{Figures/Picture4.png}
    \caption{Frequency response of the band-stop filter.}
    \label{fig:Band_Stop_Filter_Response}
\end{figure}

\newpage
\restoregeometry
\newgeometry{margin=1.2in, left=1.5in, bottom=0cm}
\section{Discussion:}
\begin{enumerate}
    \item The experiment was conducted to study the frequency response of a band-stop filter.
    \item The data from simulation, and experiment show some discrepancies.
    \item Differences in theoretical values can be attributed to ideal assumptions not accounting for parasitic elements and component tolerances in practical setups.
    \item Variations in experimental results could be due to temperature changes affecting component values, measurement inaccuracies, and signal generator limitations.
    \item To minimize errors, the experiment should be conducted in a controlled environment, ensuring stable temperature and calibrated equipment.
\end{enumerate}

\end{document}
