\documentclass[a4paper, 14pt]{extarticle}
\usepackage[utf8]{inputenc} % This informs LaTeX that the input text is encoded in UTF-8
\usepackage[margin=1.2in, left=1.5in, includefoot]{geometry} % is used to marginalize the document
\usepackage{fancyhdr} % used to change page style

% media edit
\usepackage{graphicx} % Required for inserting images
\usepackage{caption}
\usepackage{subcaption}
\usepackage{float} % allows for control of float positons of the image 
\usepackage{booktabs} % is used to improve the quality of the tables 

% text edit
\pagestyle{fancy} % to change the page style and get that overline 
\fancyhead[]{} % to delete the header over the overline
\fancyfoot[]{} % to delelte the page number from the bottom of the page
\fancyhead[R]{\thepage} % to put the page number in the right side of the overline.
\geometry{headsep=20pt}  % to make a gap of 20 pt between overline and the text

% start of the document 
\begin{document}

    \section{Objectives:}
        \begin{itemize}
            \item To determine the value of a resistor using multimeter.
            \item To compare the measured value with colour chart.
            \item To determine if a transistor is n-p-n or p-n-p.
            \item To determine the emitter, base and collector terminal of a transistor.
        \end{itemize}

    \section{Apparatus:}
        \begin{itemize}
            \item Multimeter
            \item Resistor
            \item n-p-n transistor 
            \item p-n-p transistor
            \item Breadboard
        \end{itemize}

    \newpage

    \section{Result Analysis:}
        \subsection{Transistor analysis:}
        \vspace{20pt}
            \begin{figure}[h]
                \centering
                \begin{subfigure}[h]{0.4\textwidth} %npn pointed
                    \includegraphics[width=\textwidth, height = 1.4in]{2N3904pinout.png}
                    \caption{2N3904 n-p-n transistor with Emitter, Base, Collector pointed.}
                    \label{fig:subfig1}
                \end{subfigure}
                \hfill
                \begin{subfigure}[h]{0.4\textwidth} %npn lab
                \centering
                    \includegraphics[width=0.7\textwidth, height = 1.4in]{2N3904.png}
                    \caption{2N3904 n-p-n transistor of electronics lab.}
                    \label{fig:subfig2}
                \end{subfigure}
                \vspace{20pt} 
                \begin{subfigure}[h]{0.4\textwidth} %pnp pointed
                    \includegraphics[width = \textwidth, height = 1.9in]{2N3906pinout.png}
                    \caption{2N3906 p-n-p transistor with Emitter, Base, Collector pointed.}
                    \label{fig:subfig3}
                \end{subfigure}
                \hfill
                \begin{subfigure}[h]{0.4\textwidth} %pnp lab
                    \includegraphics[width = \textwidth, height = 1.9in]{2N3906.png}
                    \caption{2N3906 p-n-p transistor of electronics lab.}
                    \label{fig:subfig4}
                \end{subfigure}
                \vspace{10pt}
                \caption{n-p-n and p-n-p transistors}
                \label{mainfig1}
            \end{figure}
            From the above figures it can be said that-
            \begin{enumerate}
                \item Transistor has three terminals
                \item If the flat side of the transistor is faced towards the student then the left most terminal is emitter, then on the right is base and then collector.
            \end{enumerate}
            
    \newpage
    
        \subsection{Transistor voltage analysis:}
            \begin{figure}[h]
                \centering
                \begin{subfigure}[h]{0.4\textwidth} %npn eb voltage
                    \includegraphics[width=\textwidth, height = 5cm]{2N3904Volatage01.png}
                        \caption{Emitter-Base voltage of n-p-n transistor.}
                        \label{fig:subfig5}
                \end{subfigure}
                \hfill
                \begin{subfigure}[h]{0.4\textwidth} %npn cb voltage
                \centering
                    \includegraphics[width=0.6\textwidth, height = 5cm]{2N3904Volatage02.png}
                        \caption{Collector-Base voltage of n-p-n transistor.}
                        \label{fig:subfig6}
                    \end{subfigure}
                \vspace{25pt} 
                \begin{subfigure}[h]{0.4\textwidth} %pnp eb voltage
                    \centering
                    \includegraphics[width = 0.7\textwidth, height = 5cm]{2N3906voltage01.png}
                    \caption{Emitter-Base voltage of p-n-p transistor.}
                    \label{fig:subfig7}
                \end{subfigure}
                \hfill
                \begin{subfigure}[h]{0.4\textwidth} %pnp cb voltage
                    \includegraphics[width = \textwidth, height = 5cm]{2N3906voltage02.png}
                    \caption{Collector-Base voltage of p-n-p transistor.}
                    \label{fig:subfig8}
                \end{subfigure}
                \vspace{5pt}
                \caption{Terminal voltage testing of transistor.}
                \label{fig:mainfig2}
            \end{figure}
            Here-
            \begin{enumerate}
                \item From \ref{fig:subfig5} and \ref{fig:subfig7} it can be seen that the Emitter-Base voltage of n-p-n transistor is 821V and p-n-p transistor is 808V.
                \item From \ref{fig:subfig6} and \ref{fig:subfig8} it can be seen that the Collector-Base voltage of n-p-n transistor is 841V and p-n-p transistor is 826V.
            \end{enumerate}
            So according to the tests it can be said that the Emitter-Base voltage is smaller compared to Collector-Base voltage.

    \newpage

        \subsection{Resistor analysis:}
            \begin{figure}[h]
            \centering
                \includegraphics[width= \textwidth]{Resistor_color_codes.png}
                \caption{Colour codes of resistors.}
                \label{fig:mainfig3}
            \end{figure}
            According to \ref{fig:mainfig3} it can be seen that in a resistor the 1st and 2nd colour represents the 1st two digits of the resistance and the 3rd colour represents the power of the 10 by which the other two digits will be multiplied. The 4th colour represents the tolerance of the resistor.
            
            \newpage
            
            \begin{figure}[htbp]
            \centering
                \begin{subfigure}[H]{0.3\textwidth}
                \centering
                    \includegraphics[width=\textwidth, height = 4cm]{120ohmbrownredbrown.png}
                    \caption{120$\Omega$ resistor.}
                    \label{fig:subfig9}
                \end{subfigure}
                \hfill
                \begin{subfigure}[H]{0.3\textwidth}
                \centering
                    \includegraphics[width = 1\textwidth, height = 4cm]{120ohmcodebrownredbrown.png}
                    \caption{120$\Omega$ resistor colour code resistance.}
                    \label{fig:subfig10}
                \end{subfigure}
                \hfill
                \begin{subfigure}[H]{0.3\textwidth}
                \centering
                    \includegraphics[width=0.7\textwidth, height = 4cm]{120ohmtestbrownredbrown.png}
                    \caption{120$\Omega$ resistor resistance test.}
                    \label{fig:subfig11}
                \end{subfigure}
                \caption{120$\Omega$ resistor test.}
                \label{fig:mainfig4}
            \end{figure}
            Here, \ref{fig:subfig9} is the resistor and \ref{fig:subfig10} shows the theoretical calculation of the resistor which is 120$\Omega$ and \ref{fig:subfig11} is the measured value of the resistor that was found to be 120$\Omega$.

            \vspace{3cm}
            
            \begin{figure}[htbp]
            \centering
                \begin{subfigure}[h]{0.3\textwidth}
                \centering
                    \includegraphics[width=\textwidth, height = 4cm]{470ohmyellowvioletbrown.png}
                    \caption{470$\Omega$ resistor.}
                    \label{fig:subfig12}
                \end{subfigure}
                \hfill
                \begin{subfigure}[H]{0.3\textwidth}
                \centering
                    \includegraphics[width = 1\textwidth, height = 4cm]{470ohmcodeyellowvioletbrown.png}
                    \caption{470$\Omega$ resistor colour code resistance.}
                    \label{fig:subfig13}
                \end{subfigure}
                \hfill
                \begin{subfigure}[h]{0.3\textwidth}
                \centering
                    \includegraphics[width=0.7\textwidth, height = 4cm]{470ohmtestyellowvioletbrown.png}
                    \caption{470$\Omega$ resistor resistance test.}
                    \label{fig:subfig14}
                \end{subfigure}
                \caption{470$\Omega$ resistor test.}
                \label{fig:mainfig5}
            \end{figure}
            Here, \ref{fig:subfig12} is the resistor and \ref{fig:subfig13} shows the theoretical calculation of the resistor which is 470$\Omega$ and \ref{fig:subfig14} is the measured value of the resistor that was found to be 462$\Omega$.

            \newpage
            
            \begin{figure}[htbp]
            \centering
                \begin{subfigure}[h]{0.3\textwidth}
                \centering
                    \includegraphics[width=\textwidth, height = 4cm]{560ohmgreenbluebrown.png}
                    \caption{560$\Omega$ resistor.}
                    \label{fig:subfig15}
                \end{subfigure}
                \hfill
                \begin{subfigure}[H]{0.3\textwidth}
                \centering
                    \includegraphics[width = 1\textwidth, height = 4cm]{560ohmcodegreenbluebrown.png}
                    \caption{560$\Omega$ resistor colour code resistance.}
                    \label{fig:subfig16}
                \end{subfigure}
                \hfill
                \begin{subfigure}[h]{0.3\textwidth}
                \centering
                    \includegraphics[width=0.7\textwidth, height = 4cm]{560ohmtestgreenbluebrown.png}
                    \caption{560$\Omega$ resistor resistance test.}
                    \label{fig:subfig17}
                \end{subfigure}
                \caption{560$\Omega$ resistor test.}
                \label{fig:mainfig6}
            \end{figure}
            Here, \ref{fig:subfig15} is the resistor and \ref{fig:subfig16} shows the theoretical calculation of the resistor which is 560$\Omega$ and \ref{fig:subfig17} is the measured value of the resistor that was found to be 546$\Omega$.

    \newpage
    
    \section{Discussion:}
        \begin{enumerate}
            \item The motive of the experiment was to measure the value of the resistor using color code with bare eyes and via multi meter, also to check if the resistor is n-p-n or p-n-p and also to differentiate between Emitter, Base, Collector.
            \item The terminals were determined by the difference in voltage.
            \item The values of the resistors measured by multimeter were a little different from the colour chart and it was determined that the change was due to mechanical error.
        \end{enumerate}
    
\end{document}
